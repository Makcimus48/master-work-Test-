\chapter*{Введение} 
\addcontentsline{toc}{chapter}{Введение}

Одним из явлений глобализации – это появления высоконагруженных систем (так же называемыми highload application / HLA). Социальные сети, банковские системы, обрабатывающие миллиарды транзакций в сутки, мессенджеры и другие системы существуют, чтобы удовлетворять потребность растущего населения. А такие приложения обусловлены высокой отказоустойчивостью и пропускной способностью.

\textbf{Актуальность} данной темы состоит в высоком спросе со стороны организаций, поддерживающие функциональность своего бизнеса, и для людей продвигающие свои стартапы, рассчитанные на широкий охват аудитории.

%применении алгоритмов поиска сходств к электронным письмам, с целью их классификации на базе синтаксической структуры предложения.

\textbf{Новизна} заключается в комплексности подхода, при разработке высоконагруженных систем в веб-приложениях. Подход обеспечивает разработку систем на основе количественных и качественных показателях, удовлетворяющих требованиям надёжности, предоставляя лучший выбор в соответствии с установленными требованиями.

% ЗДЕСЬ КАКАЯ-ТО ПУТАНИЦА.
%Поиск сходств на уровне слов является основой для поиска на уровне предложений и текстов. В зарубежной литературе выделяют лексическое и синтаксическое сходство слов. Под лексическим значением слова понимается символьная последовательность. Следовательно, лексическое сходство двух слов -- есть сходство их символьной последовательности букв. В то время как под синтаксическим значением понимается смысловое значение слова, а под синтаксическим подобием понимается синонимия слов, то есть сходство их смыслового значения \cite{mihalcea2006corpus}. С точки зрения русского языка, лексическое значение понимается как содержание слова, отображающее в сознании и закрепляющее в нем представление о предмете, свойстве, явлении, процессе и т.д.

\textbf{Цель работы} --- формирование общей методики разработки высоконагруженных систем в веб-приложениях.

Для достижения поставленной цели, выделены следующие \textbf{задачи} дипломной работы: 
\begin{enumerate}[1.]
	\item Исследование источников определения высоконагруженных веб-приложений;
	\item Анализ источников по предметной области;
	\item Формулирование требований к высоконагруженным приложениям;
	\item Анализ имеющихся подходов и средств построения высоконагруженных веб-приложений;
	\item Выделение важных критериев высоконагруженных систем, разработка тестов для оценки;
	\item Оценка накопленных экспериментальных данных статистическими методами;
	\item Проектирование и разработка веб-приложения в соответствии с результатами тестов.
\end{enumerate}

\textbf{Методы исследования.} Данная работа реализована с помощью метода научного исследования (design science research - DSR). Благодаря ему, производится чёткая идентификация и обоснование проблемы, определение целей, ограничений и требований для проектирования и дальнейшей разработки артефакта – методика разработки высоконагруженных систем в веб-приложениях. Далее производится оценка артефакта со стороны эффективности и результативности, а так же других критериев. В конце, на его основе, разрабатывается веб-приложение, проходящее апробацию.  

Для оценки экспериментальных данных используются методы математической статистики.