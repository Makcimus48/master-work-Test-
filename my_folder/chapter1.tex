\chapter{ВЫСОКОНАГРУЖЕННЫЕ ВЕБ-ПРИЛОЖЕНИЯ} \label{ch1}


\section{Highload приложения} \label{ch1:relatedSystems} 

Современный человек уже не может представить свою жизнь без всевозможных сервисов, которые облегчают его проживание. А в свою очередь эти сервисы должны обеспечивать свои услуги всем желающим. Востребованность к определённым сервисам, среди пользователей, только увеличивается: мессенджеры, которые обрабатывают сообщения миллионов пользователей ежесекундно; банки, которые обслуживают потоки транзакций своих и не только клиентов; карты с актуальной информацией о загруженности трафика и так далее. Но так же появляются новые сервисы, предоставляющие свои услуги, которые так или иначе можно считать примерами highload приложения.

Так сложилось, что зачастую под высоконагруженной системой, люди представляют в основном веб-сервис в частности какой-нибудь многопользовательский интернет-сайт. Однако это лишь часть, к ним так же относятся всевозможные управляющие бизнес процессом и регулирующие системы.

Из-за специфики понятия highload сложно провести классификацию, так как нет такого понятия как «сайт/ сервис средней нагрузки». Каждый сервис/ сайт – специфичен. Для разных систем, равное количество запросов, приводит к разным нагрузкам на разные ресурсы. Однако любой сервис highload обладает одним или несколько критериев: \cite{souders2008high}

\begin{itemize}
	\item Большой поток единовременных пользователей;
	\item Наличие сложных многоуровневых/ многочисленных расчётов и вычислений;
	\item Большой объём обрабатываемых данных.
\end{itemize}

Традиционные качества highload систем: \cite{ge2009powerpack}

\begin{itemize}
	\item \textbf{Многопользовательность.} Высоконагруженные приложения направлены на потребителей. Следовательно в единицу времени они работают с более чем 1 человеком. Десятки, тысячи, сотни тысяч пользователей могут единовременно взаимодействовать, а предсказать верхний порог – невозможно. Однако заметим, что большое количество пользователей не гарантирует высокую нагрузку на саму систему.
	\item \textbf{Распределённость.} Зачастую высоконагруженные системы распределены на несколько узлов. Необходимость в распределении возникает в следствии возрастающего объёма обрабатываемых данных или необходимости отказоустойчивости системы, чтобы приложение продолжало работать даже при отказе некоторых узлов. Заметим, что данные узлы могут располагаться как и на одной машине, когда создаются виртуальные образы узлов, так и на разных машинах.
	\item \textbf{Интерактивность.} В ключе высоконагруженных систем определяется способностью системы отвечать на входящие запросы пользователей за приемлемое время.
\end{itemize}

В качестве основных направлений высоконагруженных приложений можно выделить следующие: \cite{souders2008high, Nodejs82:online}

\begin{itemize}
	\item Мессенджеры. Один и самых распространённых примеров, где огромное число пользователей ежедневно общаются друг с другом, создавая лавину сообщений для сервиса. А так же на базе мессенджеров создаются целые корпоративные платформы для ведения бизнеса и не только. Наиболее часто упоминаемые представителями являются: Telegram, WhatsApp, Viber, Skype, Discord и так далее;
	\item Социальные сети. Аналогично как у с мессенджеров, но только разнообразие и количество контента больше чем у мессенджеров. Представителями являются: Facebook, Twitter, Instagram, Vk, Одноклассники и другие;
	\item Онлайн коммерция. Примеры различные интернет магазины такие как: Amazon, eBay и другие; или онлайн биржи;
	\item Образование. Образовательные площадки такие как Coursera, Edx И другие;
	\item Игровая индустрия. Сервера популярных площадок дистрибьюторов игр и предоставления серверов для онлайн игр. Примеры: Steam, Epic Games Store, Battle.net и другие;
	\item другие.
\end{itemize}

\section{Основные уязвимости веб-приложений} \label{ch1:majorVulnerabilities} 

Все веб-приложения подвержены стандартному набору угрозам \cite{wichers2017owasp, Securing91:online}, которые хоть и известны (достаточно редко появляется новый вид уязвимостей), но они постоянно усовершенствуются злоумышленниками. Атаки в обход аутентификации для не защищённых URL адресов, перехват данных администратора, инъекции в базе данных (SQL-инъекции), cross site request forgery (CSRF), использование особенностей языков программирования, cross site scripting (XSS), небезопасное хранение важных данных и так далее.


Согласно открытому проекту обеспечения безопасности веб-приложений (Open Web Application Security Project - OWASP) за 2017 год определён следующий список уязвимостей для веб-проектах \cite{wichers2017owasp}:
\begin{itemize}
	\item Инъекции. Угроза внедрения, такие как SQL, NoSQL, OS и LDAP инъекции, возникают, когда посылаемые данные отправляются интерпретатору, в котором нет проверки и или экранирования потенциально опасных символов. Внедрённый код злоумышленника может заставить интерпретатора выполнить непреднамеренные команды или предоставить доступ к данным без надлежащей авторизации.
	\item Сломанная аутентификация. Фоновые функции, отвечающие за аутентификацию и управлением сеансами, часто реализуются неправильно, что позволяет злоумышленникам получить пароли, ключи или токены сеансов, а так же использовать другие недостатки реализации, чтобы временно или постоянно принимать идентификационные данные других пользователей.
	\item Конфиденциальные данные (Sensitive Data Exposure). Некоторые веб-приложения и API не защищают должным образом конфиденциальные данные. Злоумышленники могут перехватить их и модифицировать для совершения мошенничества с кредитными картами, кражи личной информации или других незаконных действий. Конфиденциальные данные могут быть скомпрометированы без дополнительной защиты, что требуют специальных мер предосторожности при взаимодействии с браузером.
	\item Внешние объекты XML (XXE). Многие старые или плохо настроенные процессоры XML анализирует внешние ссылки в документах XML. Таким образом сокрытый вредоносный код в XML исполняется, предоставляя злоумышленнику доступ или конфиденциальные данные.
	\item Broken Access Control. Организация ограничения на то, что разрешено делать аутентифицированным пользователям, часто не соблюдаются должным образом. Злоумышленники могут использовать эти недостатки для доступа к несанкционированным функциям и / или данным, таким как доступ к учетным записям других пользователей, просмотр конфиденциальных файлов, изменение данных других пользователей, изменение прав доступа и т. д.
	\item Неверная конфигурация безопасности (Security Misconfiguration). Неправильная настройка безопасности является часто встречающейся проблемой. Обычно это результат оставления конфигураций по умолчанию, открытого облачного хранилища, неправильно настроенных заголовков HTTP и не настроенный обработчик ошибок, выдающий конфиденциальную информацию при ошибках. Так же важно своевременное операционные системы, платформы, библиотек и приложения.
	\item Межсайтовый скриптинг XSS (Cross-Site Scripting). Внедрение вредоносного кода на выбранную страницу с целью взаимодействия внедрённого кода с сервером злоумышленников при открытии страницы.
	\item Небезопасная десериализация (insecure deserialization). Некоторые приложения восстанавливают данные из битовой последовательности (десерализация), тем самым создавая уязвимость. Злоумышленники могут подделывать эти последовательности, тем самым может изменить логику приложения, произведя удалённое выполнение кода, подделав структуру и объекты данных или же сфальсифицировать данные.
	\item Использование компонентов с известными уязвимостями. Компоненты, такие как библиотеки и другие программные модули, обладают и исполняются с теми же привилегиями, что и само приложение. Если в компоненте заложена уязвимость, то это предоставляет доступ к полному контролю сервера.
	\item Недостаточный сбор логов и мониторинга (Components With Known Vulnerabilities). Отсутствие грамотной системы логирования и мониторинга за приложением, не корректно формирует систему реагирования на инциденты. Таким образом, предоставляя злоумышленникам проводить «разведку» с дальнейшей атакой на систему.
\end{itemize}

В табл. \ref{listOfRisk} представлены методы борьбы против ранее перечисленных угроз.

\begin{table}
	\caption{Список угроз с основными методами противодействия им}
	\label{listOfRisk}
	\begin{tabularx}{\linewidth}{|p{4cm}|X|}
		\hline
		Угроза & Метод защиты \\
		\hline
		SQL injection & Формирование SQL запросов с экранированием потенциально опасных символов. \\
		\hline
		Сломанная аутентификация & Строгое формирование маршрутной карты совместно с допустимыми разделами приложения. \\
		\hline
		Sensitive Data Exposure & Хранение личных данных должно осуществляться строго в зашифрованном виде. Формирование системы доступа к данным по ключам. \\
		\hline
		XXE & Настроить процессор XML для использования локального статического Document Type Definition (DTD)[7] и запретить любое объявленное DTD, включенное в документ XML. \\
		\hline
		Broken Access Control & Запрещение доступа к функциональности по умолчанию. Использовать списки контроля доступа на основе ролей. \\
		\hline
		Security Misconfiguration & Отключение интерфейса администрирования и режим отладки приложения. Настройка сервера для предотвращения несанкционированного доступа, просмотра каталога и т. д. \\
		\hline
		Cross-Site Scripting & Фильтровать вводимые данные пользователя на основе ожидаемых. Использовать соответствующие заголовки ответа. Чтобы запретить XSS в ответах HTTP, которые не должны содержать HTML или JavaScript. \\
		\hline
		Insecure Deserialization & Не принимать сериализованные объекты из ненадежных источников или использовать среды сериализации, которые допускают только примитивные типы данных \\
		\hline
		Components With Known Vulnerabilities & Использовать специальные программы для контроля версий компонент и поиска уязвимостей. \\
		\hline
		Insufficient Logging \& Monitoring & Подготовить план реагирования на инциденты. Реализовать систему создания журналов событий. \\
		\hline
	\end{tabularx}
\end{table}

На основе представленных угроз и способов борьбы с ними можно сформулировать общие требования к высоконагруженным сервисам:
\begin{itemize}
	\item Спроектированная система должна иметь строгую архитектуру приложения;
	\item Потоки данных – определены;
	\item Использование библиотек и фреймворков предоставляют лучшую практику;
	\item Вводимые данные (пароли) должны храниться в зашифрованном виде;
	\item Вся информация вводимая пользователем должна фильтроваться и экранироваться;
	\item Разграничение разделов приложения, согласно ролям и спроектированной маршрутизации;
	\item SQL-запросы, предварительно, должны быть подготовлены.
\end{itemize}

\section{Выводы} \label{ch1:conclusion}

В данной главе была произведена классификация highload приложений и они были определены. Были представлены направления, в которых используется высоконагруженные системы и примеры таких систем. Были проанализированы актуальные угрозы веб-приложений, а так же методы борьбы с ними.
