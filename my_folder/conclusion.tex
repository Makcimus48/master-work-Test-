\chapter*{Заключение} \label{ch-conclusion}
\addcontentsline{toc}{chapter}{Заключение}	% в оглавление 

В первой главе была произведена классификация highload приложений и они были определены. Были представлены направления, в которых используется высоконагруженные системы и примеры таких систем. Были проанализированы актуальные угрозы веб-приложений, а так же методы борьбы с ними.

Во второй главе были определены тип веб-приложения, ключевые ресурсы веб-приложений и виды хостинга. Кроме того, были проанализированы виды транспорта передачи данных с описанием достоинства и недостатков каждого. Были определены инструменты для построения таких веб-приложений. Было проведено исследование задержки реакции системы при разработке используя стандартный протокол WebSocket, Socket.io и SockJS. Проведён эксперимент по замеру времени передачи сообщения. 

В данной главе была сформулирована методика, предполагающая выполнение последовательности действий для сокращения угроз. Так же был определен стек для разработки прототипа приложения. Прототип специализированного веб-приложения для формирования отчётов по данным с заездов парусных яхт в период тренировок был разработан. Так же данный прототип подвергся нагрузочному тестированию. Были созданы тесты, которые имитируют посещение веб-приложения виртуальными пользователями. Нагрузочное тестирование показало, что прототип уступает по показателям, однако результаты можно считать сопоставимыми с результатами коммерческие аналогов, из чего следует, что предложенный архитектурный подход можно считать приемлемым.

На основании данных результатов, можно утверждать, что прототип веб-приложения, реализованный в рамках методики может составить конкуренцию аналогичным коммерческим продуктам, так как  является более предпочтительным по некоторым аспектам, например, из-за цены такого подхода.

